%-------------------------------------------------------------------------------
%	SECTION TITLE
%-------------------------------------------------------------------------------
\cvsection{Team Projects}


%-------------------------------------------------------------------------------
%	CONTENT
%-------------------------------------------------------------------------------
\begin{cventries}

% FIRST POINT
 \cventry
    {Full Stack Developer - "CSE 442 Software Development"} % Job title
    {University at Buffalo BuyNSell Website} % Organization
    {Buffalo, New York} % Location
    {Feb. 2019 - May 2019} % Date(s)
    {
      \begin{cvitems} % Description(s) of tasks/responsibilities
      {\textbf{GitHub URL}   -  \href{https://github.com/JacobYim/CSE442-BuyNSell}{JacobYim/CSE442-BuyNSell}}
        \item\item {Implemented middleware functions to capture and verify user sign-up data client-side before final push serverside to the \textit{PostgreSQL} database.}
        \item {Managed and organized code by transitioning all \textit{HTML} files into \textit{EJS} modules to further promote modularity within the entire project.}
        \item {Ensured user protection from server-side attacks by arranging password hashing to occur client-side prior to pushes to the \textit{PostgreSQL} database.}
        \item {Provisioned system for user to reset password using the \textit{nodemailer API} to mail a temporary password based on the user's inputted email.}
      \end{cvitems}
    }
    
% SECOND POINT
\cventry
    {Project Manager - "CSE 474 Machine Learning"} % Job title
    {Number Classification using Neural Networks} % Organization
    {Buffalo, New York} % Location
    {Mar. 2019 - Apr. 2019} % Date(s)
    {
      \begin{cvitems} % Description(s) of tasks/responsibilities
      {\textbf{GitHub URL}   -  \href{https://github.com/ltanedo/CSE-474-Assignment-2}{ltanedo/CSE-474-Assignment-2}}
        \item\item {Implemented all matrix math formulas for forward and back propagation from lecture into \textit{Python} using the \textit{Numpy} library and its functions.}
        \item {Identified Google’s \textit{TensorFlow} to be more efficient than \textit{Python} scripts using \textit{Numpy} when processing large data sets using a \textit{Neural Network}.}
        \item {Built \textit{Neural Network} using class provided \textit{API} to create and train provided data using the given examples with provided expected outputs.}
      \end{cvitems}
    }

%---------------------------------------------------------
\end{cventries}