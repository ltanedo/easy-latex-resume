%-------------------------------------------------------------------------------
%	SECTION TITLE
%-------------------------------------------------------------------------------
\cvsection{University Work}


%-------------------------------------------------------------------------------
%	CONTENT
%-------------------------------------------------------------------------------
\begin{cventries}

\cventry
  {Full-Stack Developer - under ph.d candidate Abhishek Gupta} % Job title
  {Collaboration with CarbinAI (MIT-Group)} % Organization
  { ltanedo/EasySensorTools-Android } % Location
  {Jan. 2022 - Aug. 2022} % Date(s)
  {
    \begin{cvitems} % Description(s) of tasks/responsibilities
      \item {Worked with Ubuffalo and MIT teams, to identify bugs and improvements to the original CarbinApp and CollectionSDK in \textit{Kotlin}.}
      \item {Re-designed testing app for CarbinAI, added the group's newer in-house sdk, also widening the device compatability range from \textit{Android} 8-12.}
      \item {Created \textit{EasySensorTools}, an sdk to easily collect sensor data with support for \textit{Protobuf Streaming}, \textit{AWS Integration}, and Backend analysis Tools.  }
    \end{cvitems}
  }

\cventry
  {Full-Stack - Developer under Dr. Zhen Liu and Dr. Chunming Qiao} % Job title
  {WizardsOfOdds Trading-Infrastructure Project} % Organization
  { ltanedo/easy-algotrading-sdk } % Location
  {Oct. 2021 - Aug. 2022} % Date(s)
  {
    \begin{cvitems} % Description(s) of tasks/responsibilities
      \item { Worked with UBIT and former traders, enabled Flask as a service to all UB-Students and designed a Trading-Companion App in React-Native.}
      \item { Using algorithmic complexity principals, reduced time of original recommendation algorithm from roughly 10 minutes to 30 seconds. }
      \item { Created \textit{easy-algo-trading-sdk}, an sdk to easily schedule trades and recommendations using caching and \textit{gsheets} for fast-executions. }
    \end{cvitems}
  }

  \cventry
  {Back-end Developer - under Dr. Chunming Qiao} % Job title
  {(NYDOT) Incident-Inspection-Management-Service : Buffalo Team} % Organization
  { ltanedo/nyc-point-to-road-labeler } % Location
  {Aug. 2021 - Dec. 2021} % Date(s)
  {
    \begin{cvitems} % Description(s) of tasks/responsibilities
      \item { Worked with RIT Road Visualization lab, to develop a global road index for easy lookup, based on existing \textit{TMC} identifier codes. }
      \item { Worked with Tianyu Bao (UB Data-Science Major), to combine RIT-TMC table and vehicle incident Meta-Data to final-matched output. }
      \item { Reduced incident-matching time from 10~minutes to 10 seconds, used \textit{kd-j tree} from \textit{sk.learn} (\textit{Python}) to reduce lookup to O(log(n)) time.}
    \end{cvitems}
  }

  \cventry
  { Under Dr. Matthew Hertz } % Job title
  { Teaching Assistant for CSE 442 Senior Design } % Organization
  { CSE-Courses/... } % Location
  { Aug. 2021 - Dec. 2021 } % Date(s)
  {
    \begin{cvitems} % Description(s) of tasks/responsibilities
      \item { Worked with two Senior-Design teams to release MVP (minimum-viable products) Web-Apps for four seperate sprints (grading periods). }
      \item { Identified skills and weaknesses of each team member, efficiently assign tasks based on said skills to avoid merge conflicts.}
      \item { Tracked and monitored performance of both teams using \textit{Agile Development} and \textit{Zenhub} identifying task issues and completions.}
    \end{cvitems}
  }

  \cventry
  { Data-Scientist under Dr. Lu Su and Dr. Weida Zhong} % Job title
  { UB PocketCare+ (Pre-PocketCareS) } % Organization
  { PocketCareS/Android } % Location
  { Jan. 2020 - March. 2020 } % Date(s)
  {
    \begin{cvitems} % Description(s) of tasks/responsibilities
      \item { During Covid-Pandemic, worked with UB-Team to upgrade older report-health app from \textit{Android} 4.0, to \textit{Android} 10.0.}
      \item { Personally advocated for design choices such as button placements and visual feedbacks to promote accessible design. }
      \item { Upgraded base app to use UB's primary and secondary HEX colors also implementing Modern Material Design to all components.}
    \end{cvitems}
  }

  \cventry
  { Data-Scientist under Dr. Lu Su and Dr. Weida Zhong} % Job title
  { Road-Sensing for Smart Cities Project } % Organization
  { ltanedo/osm-snap-to-roads } % Location
  { Jun. 2019 - Aug. 2019 } % Date(s)
  {
    \begin{cvitems} % Description(s) of tasks/responsibilities
      \item { Worked with Dr. Weida Zhong (formerly ph.d candidate) to design a lower cost infrastructure pipeline when parsing data from UB-Buses. }
      \item { Saved entire research budget monetarily, migrated tech stack from \textit{Google Maps API} to \textit{Open-Street-Maps SDK} making all API calls zero-cost.}
      \item { Identified common gps outliers on bus-routes, created backend pipeline to clean and convert gps points to xml, snapping all points to roads.}
    \end{cvitems}
  }

% \cventry
%   {Full-Stack Developer - under ph.d candidate Abhishek Gupta} % Job title
%   {Collaboration with CarbinAI (MIT-Group)} % Organization
%   { ltanedo/EasySensorTools-Android } % Location
%   {Jun. 2019 - Aug. 2019} % Date(s)
%   {
%     \begin{cvitems} % Description(s) of tasks/responsibilities
%       \item {Reduced the monetary cost of map matching (per API call) to zero by implementing the transition from \textit{Google Maps} to \textit{Open Street Maps}}
%       \item {Designed process to detect and remove outliers (caused by broadcast interference) in original GPS data from the UB Stampede bus system.}
%       \item {Provisioned system to pre-process all GPS files in the \textit{‘.txt’} format into the \textit{‘.gpx’} format to ensure compatibility with the \textit{Open Street Maps API}.}
%       \item {Built feature to identify the average speed for all GPS files per a user's given interval also writing the results to local storage for future lookup.}
%       \item {Created web app to map-match and display user-inputted GPS data highlighting the path based on average speed taken from all GPS files.}
%     \end{cvitems}
%   }    

%---------------------------------------------------------
\end{cventries}